\chapter{Introdução}
\section{Apresentação Geral}

A Internet das Coisas (IoT) é um paradigma onde dispositivos inteligentes (smartphones, tablets, sensores, etc) são capazes de comunicar-se uns com os outros cooperando entre si  para alcançar objetivos comuns \cite{atzori2010internet}. Estima-se que em meados de 2020 existirão mais de 24 bilhões de dispositivos inteligentes conectados causando um impacto econômico de trilhões de dólares\cite{gubbi2013internet,rivera2014gartner}. De acordo com Loghin et al.\cite{loghin2015performance}, o aumento do volume, variedade e velocidade do dados gerados por estes dispositivos requer um maior dimensionamento de recursos de DataCenter (DC), pois as características intrínsecas a este paradigma trazem diversos desafios para o desenvolvimento de aplicações que buscam acessar, integrar e analisar a quantidade de dados produzida por estes dispositivos.

Nesta direção, a computação em nuvem pode resolver estes problemas oferecendo armazenamento sob demanda e escalável, bem como serviços de processamento que podem se adaptar aos requisitos do IoT. Contudo, 
para processar o grande volume de informações, a infraestrutura tradicional baseada em nuvem pode levar a um longo tempo de resposta e maior consumo de largura de banda \cite{bonomi2014fog}. Além disso, em aplicações sensíveis a latência como por exemplo, monitoramento de saúde, veículos autônomos e sistemas de emergência, o atraso causado pela transferência de dados ou a indisponibilidade de comunicação com a nuvem pode ser inaceitável\cite{bonomi2014fog,dastjerdi2016fog}.

Para superar essas limitações, o paradigma de computação em neblina ou "Fog Computing"\cite{computing2016internet} foi proposto pela Cisco em 2012. Este conceito amplia a noção de computação em nuvem, onde os requisitos de processamento, rede e armazenamento seriam atendidos na borda da rede. Devido a esta característica, a computação de nevoeiro emergiu como uma solução promissora para permitir o processamento contínuo de dados na proximidade do usuário em tempo real.

\section{Motivação}


Para alcançar todo o potencial que os paradigmas Fog e IoT oferecem para análise de dados em tempo real, vários desafios precisam ser superados \cite{dastjerdi2016fog,computing2016internet,gupta2016ifogsim}. 

Segundo Gupta et al.\cite{gupta2016ifogsim}, o primeiro e mais crítico problema é projetar técnicas de gerenciamento de recursos que determinem quais módulos de aplicação serão apresentados para cada dispositivo na borda da rede para minimizar a latência e maximizar a taxa de transferência. Além disso, é necessário realizar estas tarefas consumindo o mínimo de energia uma vez que, o tempo de vida de alguns sistemas de tempo real é determinado por baterias. Portanto faz-se necessário a adoção de mecanismos de gerenciamento que estendam ao máximo o tempo de vida dos sistema conservando energia ao mesmo tempo em que mantém níveis aceitáveis de qualidade de serviço (QoS). 

\section{Objetivo Geral}

\subsection{Objetivos Específicos}

\section{Metodologia}

A primeira parte desta dissertação consiste em realizar uma revisão bibliográfica para poder analisar o estado da arte do tema proposto a partir da literatura disponível, e com isso ter um embasamento teórico necessário para o desenvolvimento deste trabalho.

\section{Estrutura do Documento}

Para facilitar a navegação e melhor entendimento, este documento está
estruturado em capítulos e seções, que são:
\begin{itemize}
\item {Capítulo 1 - Introdução}: \cite{Yu:2004:ESG:1015090.1015207};
\item {Capítulo 2 - Conceitos Básicos}: \cite{Cormen:2009};
\item {Capítulo 3 - Estado da Arte}: \cite{Weicker:1984:DSS:358274.358283}
\item {Capítulo 4 - Trabalho Propost}o: \cite{IEEE_802_11:6178212};
\item {Capítulo 5 - Resultados}: \cite{Linux:402081};
\item {Capítulo 6 - Conclusão}: \cite{SBC:2012};
\end{itemize}
