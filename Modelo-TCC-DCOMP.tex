% ------------------------------------------------------------------------
% ------------------------------------------------------------------------
% abnTeX2: Modelo de Trabalho Acadêmico em conformidade com 
% as normas da ABNT
% ------------------------------------------------------------------------
% ------------------------------------------------------------------------

\documentclass[english, 
               brazil, 
               bsc] %Opções bsc (TCC) e msc (Mestrado)
               {dcomp-abntex2}

% Geração de dummy text
% Retirar para a versão final do documento
\usepackage{lipsum}


%Compila o indice
\makeindex

\begin{document}

% Seleciona o idioma do documento (conforme pacotes do babel)
\selectlanguage{brazil}

% Retira espaço extra obsoleto entre as frases.
\frenchspacing 

% ----------------------------------------------------------
% ELEMENTOS PRÉ-TEXTUAIS
% ----------------------------------------------------------
\pretextual

\titulo{Utilizando Algoritmos Mágicos para Resolver Problemas de Bancos de dados Obscuros em Nuvens Cumulonimbus}
\autor{Arnold Schwarzenegger da Silva}
\orientador{Andrew S. Tanenbaum}
\coorientador{Donald Knuth}
\curso{Ciência da Computação}

\imprimircapa
\imprimirfolhaderosto*

   
\begin{dedicatoria}
   \vspace*{\fill}
   \centering
   \noindent
   \textit{I dedicate this thesis to all my family, friends and \\ 
   professors who gave me the necessary support to get here.} \vspace*{\fill}
\end{dedicatoria}
% ---
\begin{agradecimentos}

\lipsum[1-4]

\end{agradecimentos}
% ---
\begin{epigrafe}[]
    \vspace*{\fill}
	\begin{flushright}
	
		\textit{Este trabalho, além de cultural, filosófico e pedagógico\\
				É também medicinal, preventivo e curativo\\
				Servindo entre outras coisas para pano branco e pano preto\\
				Curuba e ferida braba\\
				Piolho, chulé e caspa\\
				Cravo, espinha e berruga\\
				Panarismo e água na pleura\\
				Só não cura o velho chifre\\
				Por que não mata a raiz\\
				Pois fica ela encravada\\
				No fundo do coração\\
				(Falcão)}
		
	\end{flushright}
\end{epigrafe}
% ---
% resumo em português
\setlength{\absparsep}{18pt} % ajusta o espaçamento dos parágrafos do resumo
\begin{resumo}
 
A prática cotidiana prova que a expansão dos mercados mundiais não pode mais se dissociar do processo de comunicação como um todo. Todas estas questões, devidamente ponderadas, levantam dúvidas sobre se a necessidade de renovação processual ainda não demonstrou convincentemente que vai participar na mudança das diversas correntes de pensamento. As experiências acumuladas demonstram que a adoção de políticas descentralizadoras acarreta um processo de reformulação e modernização das novas proposições. 

Gostaria de enfatizar que a determinação clara de objetivos apresenta tendências no sentido de aprovar a manutenção das condições inegavelmente apropriadas. O cuidado em identificar pontos críticos no aumento do diálogo entre os diferentes setores produtivos talvez venha a ressaltar a relatividade do fluxo de informações. Pensando mais a longo prazo, a complexidade dos estudos efetuados garante a contribuição de um grupo importante na determinação do sistema de participação geral. O que temos que ter sempre em mente é que a revolução dos costumes estende o alcance e a importância das regras de conduta normativas. Por outro lado, a consolidação das estruturas representa uma abertura para a melhoria do orçamento setorial. 

É claro que a mobilidade dos capitais internacionais aponta para a melhoria das diretrizes de desenvolvimento para o futuro. Evidentemente, o desenvolvimento contínuo de distintas formas de atuação exige a precisão e a definição do levantamento das variáveis envolvidas.

Podemos já vislumbrar o modo pelo qual a determinação clara de objetivos representa uma abertura para a melhoria das diversas correntes de pensamento. O cuidado em identificar pontos críticos na necessidade de renovação processual estimula a padronização dos paradigmas corporativos. A prática cotidiana prova que o início da atividade geral de formação de atitudes garante a contribuição de um grupo importante na determinação do retorno esperado a longo prazo. As experiências acumuladas demonstram que a hegemonia do ambiente político apresenta tendências no sentido de aprovar a manutenção das regras de conduta normativas. 

Desta maneira, a revolução dos costumes afeta positivamente a correta previsão das direções preferenciais no sentido do progresso. Por conseguinte, a determinação clara de objetivos auxilia a preparação e a composição dos métodos utilizados na avaliação de resultados. A certificação de metodologias que nos auxiliam a lidar com o julgamento imparcial das eventualidades apresenta tendências no sentido de aprovar a manutenção dos relacionamentos verticais entre as hierarquias. 

 \textbf{Palavras-chave}: Algoritmos, Computação em Nuvem, Banco de Dados, Lero-Lero.
\end{resumo}
% resumo em inglês
\setlength{\absparsep}{18pt} % ajusta o espaçamento dos parágrafos do resumo
\begin{resumo}[Abstract]
 \begin{otherlanguage*}{english}
   
The daily practice proves that the expansion of world markets can no longer be dissociated from the communication process as a whole. All these questions, properly considered, raise questions about the need for procedural renovation has not yet demonstrated convincingly that will participate in the change of the various schools of thought. The accumulated experience shows that the adoption of decentralization policies entails a process of reform and modernization of new propositions. 

I would like to emphasize that the clear definition of objectives presents trends in order to approve the maintenance of the undeniably appropriate conditions. Care to identify critical points in increasing the dialogue between the different productive sectors may prove to emphasize the relativity of information flow. Thinking longer term, the complexity of the studies conducted ensures the contribution of an important group in determining the general participation system. What we must always keep in mind is that the revolution of morals extends the reach and the importance of rules of conduct regulations. Moreover, the consolidation of the structures is an aperture for improving the budget sector. 

It is clear that the mobility of international capital points to improve the development guidelines for the future. Of course, the continued development of different forms of performance requires precision and defining the lifting of the variables involved. 

We can already glimpse the way the clear definition of objectives is an opening for the improvement of the various schools of thought. Care to identify critical points in need of renovation procedural encourages standardization of corporate paradigms. The daily practice proves that the beginning of the general activity of forming attitudes ensures the contribution of an important group in determining the expected long-term return. The accumulated experience shows that the hegemony of the political environment presents trends in order to approve the maintenance of rules of conduct regulations. 

Thus, the revolution of customs positively affects the correct prediction of the preferred directions towards progress. Therefore, a clear definition of objectives helps the preparation and composition of the methods used in evaluating results. The certification methodologies that help us cope with the impartial judgment of eventualities presents trends in order to approve the maintenance of vertical relationships between the hierarchies.
 
   \textbf{Keywords}: Algorithms, DataBase, Cloud Computing, Lero-Lero.
 \end{otherlanguage*}
\end{resumo}
    
% Lista de Figuras
\pdfbookmark[0]{\listfigurename}{lof}
\listoffigures*
\cleardoublepage

% Lista de Tabelas
\pdfbookmark[0]{\listtablename}{lot}
\listoftables*
\cleardoublepage

% Lista de Códigos
\pdfbookmark[0]{\listlistingname}{lol}
\begin{KeepFromToc}
	\listoflistings
\end{KeepFromToc}
\cleardoublepage
   
% ---
% inserir lista de abreviaturas e siglas
% ---

\begin{siglas}
	\item[ABNT]{Associação Brasileira de Normas Técnicas}
	\item[abnTeX]{ABsurdas Normas para TeX}
  	\item[AFM]{Alphabet Frequency Matrix}
	\item[API]{Application Programming Interface}
	\item[ARIMA]{Auto-Regressive Integrated Moving Average}
	\item[BRN]{Bug Report Network}
	\item[BTS]{Bug Triage System}
	\item[CAS]{Context-Aware Systems}
	\item[CCB]{Change Control Board}
	\item[CR]{Change Request}
	\item[CVS]{Concurrent Version System}
	\item[ES]{Expert System}
	\item[FLOSS]{Free/Libre Open Source Software}
	\item[GLR]{Generalized Linear Regression}
	\item[GQM]{Goal Question Metric}
	\item[HTML]{HyperText Markup Language}
	\item[IR]{Information Retrieval}
	\item[IRT]{Recôncavo Institute of Technology}
	\item[JDT]{Jazz Duplicate Finder}
	\item[LDA]{Latent Dirichlet Allocation}
	\item[LOC]{Lines of Code}
	\item[LSI]{Latent Semantic Indexing}
	\item[MS]{Mapping Study}
	\item[MSR]{Mining Software Repositories}
	\item[NLP]{Natural Language Processing}
	\item[PROMISE]{Predictive Models in Software Engineering}
	\item[RBES]{Rule-Based Expert System}
	\item[RHEL]{RedHat Enterprise Linux}
	\item[SaaS]{Software as a Service}
	\item[SCM]{Software Configuration Management}
	\item[SERPRO]{Brazilian Federal Organization for Data Processing}
	\item[SLR]{Stepwise Linear Regression}
	\item[SLR]{Systematic Literature Review}
	\item[SVD]{Singular Value Decomposition}
	\item[SVM]{Support Vector Machine}
	\item[SVN]{Subversion}
	\item[TF-IDF]{Term Frequency-Inverse Document Frequency}
	\item[VSM]{Vector Space Model}
	\item[XP]{Extreming Programming}
\end{siglas}
% ---
% ---
% inserir lista de símbolos
% ---

\begin{simbolos}
  \item[$ \Gamma $] Letra grega Gama
  \item[$ \Lambda $] Lambda
  \item[$ \zeta $] Letra grega minúscula zeta
  \item[$ \in $] Pertence
\end{simbolos}
% ---
    
\pdfbookmark[0]{\contentsname}{toc}
\tableofcontents*
\cleardoublepage

% ----------------------------------------------------------
% ELEMENTOS TEXTUAIS
% ----------------------------------------------------------
\textual
\chapter{Introdução}

Nunca é demais lembrar o peso e o significado destes problemas, uma vez que a consolidação das estruturas é uma das consequências dos conhecimentos estratégicos para atingir a excelência. Não obstante, a contínua expansão de nossa atividade causa impacto indireto na reavaliação das posturas dos órgãos dirigentes com relação às suas atribuições. As experiências acumuladas demonstram que o aumento do diálogo entre os diferentes setores produtivos representa uma abertura para a melhoria do processo de comunicação como um todo. Evidentemente, o surgimento do comércio virtual prepara-nos para enfrentar situações atípicas decorrentes das direções preferenciais no sentido do progresso. A certificação de metodologias que nos auxiliam a lidar com a percepção das dificuldades facilita a criação dos modos de operação convencionais. 

O cuidado em identificar pontos críticos no comprometimento entre as equipes cumpre um papel essencial na formulação do retorno esperado a longo prazo.

\section{AbnTeX2}
Este documento e seu código-fonte são exemplos de referência de uso da classe
\emph{abntex2} e do pacote \emph{abntex2cite}. O documento 
exemplifica a elaboração de trabalho acadêmico (tese, dissertação e outros do
gênero) produzido conforme a ABNT NBR 14724:2011 \emph{Informação e documentação
- Trabalhos acadêmicos - Apresentação}.

A expressão ``Modelo Canônico'' é utilizada para indicar que \abnTeX\ não é
modelo específico de nenhuma universidade ou instituição, mas que implementa tão
somente os requisitos das normas da ABNT. Uma lista completa das normas
observadas pelo \abnTeX\ é apresentada em \citeonline{abntex2classe}.

Sinta-se convidado a participar do projeto \abnTeX! Acesse o site do projeto em
\url{http://www.abntex.net.br/}. Também fique livre para conhecer,
estudar, alterar e redistribuir o trabalho do \abnTeX, desde que os arquivos
modificados tenham seus nomes alterados e que os créditos sejam dados aos
autores originais, nos termos da ``The \LaTeX\ Project Public
License''\footnote{\url{http://www.latex-project.org/lppl.txt}}.

Encorajamos que sejam realizadas customizações específicas deste exemplo para
universidades e outras instituições --- como capas, folha de aprovação, etc.
Porém, recomendamos que ao invés de se alterar diretamente os arquivos do
\abnTeX, distribua-se arquivos com as respectivas customizações.
Isso permite que futuras versões do \abnTeX~não se tornem automaticamente
incompatíveis com as customizações promovidas. Consulte
\citeonline{abntex2-wiki-como-customizar} par mais informações.

Este documento deve ser utilizado como complemento dos manuais do \abnTeX\ \cite{abntex2classe,abntex2cite,abntex2cite-alf} e da classe \emph{memoir \cite{memoir}}. 

Esperamos, sinceramente, que o \abnTeX\ aprimore a qualidade do trabalho que
você produzirá, de modo que o principal esforço seja concentrado no principal:
na contribuição científica.

Equipe \abnTeX 

Lauro César Araujo

\section{Estratégias em um Novo Paradigma Globalizado}
Por conseguinte, a contínua expansão de nossa atividade apresenta tendências no sentido de aprovar a manutenção das posturas dos órgãos dirigentes com relação às suas atribuições. Por outro lado, a hegemonia do ambiente político exige a precisão e a definição do impacto na agilidade decisória. No mundo atual, o desafiador cenário globalizado facilita a criação das direções preferenciais no sentido do progresso. No entanto, não podemos esquecer que o entendimento das metas propostas estende o alcance e a importância das condições financeiras e administrativas exigidas. Pensando mais a longo prazo, a valorização de fatores subjetivos garante a contribuição de um grupo importante na determinação das regras de conduta normativas, como exemplo o Códigos \ref{labelJava} e o Código \ref{labelPython}.

\begin{listing}[H]
    \caption{Primeiro código C}
    \label{labelJava}
    
    \begin{minted}{c}
    int main() {
        printf("hello world");
        return 0;
    }
    \end{minted}
    
\end{listing}

\begin{listing}[H]
    \caption{Primeiro código Python}
    \label{labelPython}
	\begin{minted}{python}
import numpy as np
 
def incmatrix(genl1,genl2):
    m = len(genl1)
    n = len(genl2)
    M = None #to become the incidence matrix
    VT = np.zeros((n*m,1), int)  #dummy variable
 
    #compute the bitwise xor matrix
    M1 = bitxormatrix(genl1)
    M2 = np.triu(bitxormatrix(genl2),1) 
 
    for i in range(m-1):
        for j in range(i+1, m):
            [r,c] = np.where(M2 == M1[i,j])
            for k in range(len(r)):
                VT[(i)*n + r[k]] = 1;
                VT[(i)*n + c[k]] = 1;
                VT[(j)*n + r[k]] = 1;
                VT[(j)*n + c[k]] = 1;
 
                if M is None:
                    M = np.copy(VT)
                else:
                    M = np.concatenate((M, VT), 1)
 
                VT = np.zeros((n*m,1), int)
 
    return M
	\end{minted}
\end{listing}

É importante questionar o quanto a adoção de políticas descentralizadoras desafia a capacidade de equalização dos índices pretendidos. Neste sentido, a constante divulgação das informações promove a alavancagem do processo de comunicação como um todo. As experiências acumuladas demonstram que a consolidação das estruturas obstaculiza a apreciação da importância dos níveis de motivação departamental. Acima de tudo, é fundamental ressaltar que a consulta aos diversos militantes oferece uma interessante oportunidade para verificação das condições inegavelmente apropriadas. A prática cotidiana prova que o início da atividade geral de formação de atitudes acarreta um processo de reformulação e modernização do retorno esperado a longo prazo. 

Não obstante, o novo modelo estrutural aqui preconizado prepara-nos para enfrentar situações atípicas decorrentes dos paradigmas corporativos. Gostaria de enfatizar que a mobilidade dos capitais internacionais afeta positivamente a correta previsão das novas proposições. O que temos que ter sempre em mente é que o desenvolvimento contínuo de distintas formas de atuação representa uma abertura para a melhoria do investimento em reciclagem técnica. Ainda assim, existem dúvidas a respeito de como a necessidade de renovação processual talvez venha a ressaltar a relatividade dos métodos utilizados na avaliação de resultados. 

Nunca é demais lembrar o peso e o significado destes problemas, uma vez que o consenso sobre a necessidade de qualificação aponta para a melhoria do remanejamento dos quadros funcionais. A nível organizacional, o surgimento do comércio virtual maximiza as possibilidades por conta do sistema de participação geral. O empenho em analisar a crescente influência da mídia possibilita uma melhor visão global do orçamento setorial. 

Assim mesmo, a competitividade nas transações comerciais auxilia a preparação e a composição dos modos de operação convencionais. O cuidado em identificar pontos críticos no comprometimento entre as equipes é uma das consequências de alternativas às soluções ortodoxas. Percebemos, cada vez mais, que a estrutura atual da organização nos obriga à análise dos procedimentos normalmente adotados. Todavia, o julgamento imparcial das eventualidades pode nos levar a considerar a reestruturação do sistema de formação de quadros que corresponde às necessidades. 


\section{Objetivos}

Desta maneira, a expansão dos mercados mundiais desafia a capacidade de equalização das diversas correntes de pensamento. O que temos que ter sempre em mente é que a necessidade de renovação processual representa uma abertura para a melhoria das regras de conduta normativas. Nunca é demais lembrar o peso e o significado destes problemas, uma vez que a contínua expansão de nossa atividade talvez venha a ressaltar a relatividade dos modos de operação convencionais. Por conseguinte, o desenvolvimento contínuo de distintas formas de atuação auxilia a preparação e a composição do sistema de formação de quadros que corresponde às necessidades.

Pensando mais a longo prazo, a competitividade nas transações comerciais facilita a criação dos relacionamentos verticais entre as hierarquias. Caros amigos, a consulta aos diversos militantes maximiza as possibilidades por conta dos paradigmas corporativos. Assim mesmo, o surgimento do comércio virtual nos obriga à análise do retorno esperado a longo prazo.

É importante questionar o quanto a valorização de fatores subjetivos estimula a padronização das posturas dos órgãos dirigentes com relação às suas atribuições. Ainda assim, existem dúvidas a respeito de como a hegemonia do ambiente político obstaculiza a apreciação da importância das direções preferenciais no sentido do progresso. É claro que a execução dos pontos do programa garante a contribuição de um grupo importante na determinação do investimento em reciclagem técnica.

\subsection{Metodologia}

O incentivo ao avanço tecnológico, assim como a necessidade de renovação processual pode nos levar a considerar a reestruturação dos procedimentos normalmente adotados. Todavia, a constante divulgação das informações oferece uma interessante oportunidade para verificação do sistema de formação de quadros que corresponde às necessidades. No entanto, não podemos esquecer que a mobilidade dos capitais internacionais talvez venha a ressaltar a relatividade do sistema de participação geral.

Por conseguinte, a competitividade nas transações comerciais aponta para a melhoria das regras de conduta normativas. É importante questionar o quanto o fenômeno da Internet ainda não demonstrou convincentemente que vai participar na mudança dos relacionamentos verticais entre as hierarquias. Caros amigos, a execução dos pontos do programa maximiza as possibilidades por conta dos paradigmas corporativos. Assim mesmo, o aumento do diálogo entre os diferentes setores produtivos auxilia a preparação e a composição das condições inegavelmente apropriadas.

Pensando mais a longo prazo, a valorização de fatores subjetivos estimula a padronização das posturas dos órgãos dirigentes com relação às suas atribuições. Ainda assim, existem dúvidas a respeito de como a hegemonia do ambiente político obstaculiza a apreciação da importância das diversas correntes de pensamento. Acima de tudo, é fundamental ressaltar que a consulta aos diversos militantes cumpre um papel essencial na formulação dos modos de operação convencionais.

A prática cotidiana prova que a estrutura atual da organização nos obriga à análise do orçamento setorial. A certificação de metodologias que nos auxiliam a lidar com o desafiador cenário globalizado prepara-nos para enfrentar situações atípicas decorrentes das formas de ação. Evidentemente, o novo modelo estrutural aqui preconizado faz parte de um processo de gestão do investimento em reciclagem técnica. O que temos que ter sempre em mente é que o desenvolvimento contínuo de distintas formas de atuação apresenta tendências no sentido de aprovar a manutenção de todos os recursos funcionais envolvidos.

Todas estas questões, devidamente ponderadas, levantam dúvidas sobre se a determinação clara de objetivos promove a alavancagem do impacto na agilidade decisória. No mundo atual, o entendimento das metas propostas não pode mais se dissociar dos níveis de motivação departamental. Gostaria de enfatizar que o julgamento imparcial das eventualidades representa uma abertura para a melhoria do processo de comunicação como um todo. O cuidado em identificar pontos críticos no acompanhamento das preferências de consumo estende o alcance e a importância das direções preferenciais no sentido do progresso.

\section{Estrutura do Documento}

Para facilitar a navegação e melhor entendimento, este documento está
estruturado em capítulos e seções, que são:
\begin{itemize}
\item {Capítulo 1 - Introdução}: \cite{Yu:2004:ESG:1015090.1015207};
\item {Capítulo 2 - Conceitos Básicos}: \cite{Cormen:2009};
\item {Capítulo 3 - Estado da Arte}: \cite{Weicker:1984:DSS:358274.358283}
\item {Capítulo 4 - Trabalho Propost}o: \cite{IEEE_802_11:6178212};
\item {Capítulo 5 - Resultados}: \cite{Linux:402081};
\item {Capítulo 6 - Conclusão}: \cite{SBC:2012};
\end{itemize}

\chapter{Trabalhos Relacionados}

Nesta seção, identificamos e discutimos potenciais problemas no contexto da computação em nevoeiro em relação a gerenciamento de recursos e eficiência energética. Alguns deles podem ser a direção para trabalhos futuros. 

\section{Alocação de Recursos}

Kaur, Kuljeet, et al. \cite{kaur2017container}, apresentam CoESMS, uma arquitetura de software para seleção e escalonamento de tarefas na borda da rede utilizando container-as-a-service (CoaaS). Os autores propõem modelos baseados na teoria dos Jogos, a fim de minimizar a utilização total de energia dos servidores enquanto escalonam as tarefas em contêineres. A internet das coisas consiste em bilhões de dispositivos inteligentes interligados para atender demandas em áreas como saúde, transportes, agricultura, etc. Considerando o aumento na última década na adoção destes dispositivos e o volume de informações gerados de forma exponencial, faz-se necessário a adoção de métodos eficientes para processar e analisar estes dados \cite{evans2011internet}. Além disso o grande número de interconexões também estará associados ao tráfego de rede em tempo real para acessar diferentes serviços. Isso exigiria computação e processamento contínuos dos dados coletados de vários dispositivos com capacidades heterogêneas.

Métricas de desempenho, como por exemplo, poder computacional, entrega de serviços, desempenho, confiabilidade, escalabilidade e mobilidade apresentaram melhorias significativas neste contexto com avanços tecnológicos recentes [8]. Contudo, em contraste com este cenário, o tempo de resposta, custo por serviço, investimento em infra-estrutura e tempo de latência diminuíram gradualmente com o aumento dos avanços tecnológicos. Considerando estas questões, os autores adotam o conceito de Fog Computing \cite{computing2006architectural}, este conceito amplia a noção de computação em nuvem para um ambiente mais próximo ao cliente, onde os requisitos de computação, rede e armazenamento seriam atendidos na borda da rede.

Para suportar a transmissão de dados em tempo real de forma eficiente, é necessário agendar as várias tarefas de forma ideal para atender a vários parâmetros do SLA, como o tempo de resposta, a disponibilidade do serviço, a latência e os custos. Com este propósito, os autores desenvolveram uma função multi-objetivo para reduzir o consumo de energia e o makespan, ou  seja, o tempo total de processamento de todas as tarefas em todas as máquinas, considerando diferentes restrições, como memória, CPU e orçamento do usuário. 

A arquitetura proposta neste trabalho é composta por três entidades principais: consumidores de recursos, provedor de serviços públicos e fornecedores que são subdivididos em cinco camadas diferentes, apresentadas a seguir.

Resource Consumer Layer, esta camada compreende os consumidores de recursos da Fog, como clientes de TI e usuários finais, ou seja, esta camada é responsável apenas pela agregação e redirecionamento das solicitações de tarefas do usuário para o CoESMS.

Requirement Collector Layer, esta camada compreende dois módulos: o módulo de requisição de tarefa (TRM) e o módulo de solicitação de tarefa do usuário (UTRM). Usando esses dois módulos, as tarefas que seriam executadas na borda serão selecionadas, enquanto o resto das tarefas será encaminhado para o núcleo da rede.

Controller Layer, esta camada compreende dois módulos: o módulo de monitoramento de recursos (RMM) e o módulo de agendamento de contêineres (CSM). O RMM mantém continuamente o controle dos recipientes inativos que estão disponíveis para agendamento de tarefas. Com base nas informações recebidas deste módulo, o CSM escalona as tarefas nos contêineres de forma eficiente de energia usando técnicas baseadas na teoria dos jogos.

Broker Layer, Esta camada foi projetada para selecionar, classificar e reservar combinações de recursos na forma de contêineres atuando entre as camadas 3 e 5. Isto é alcançado com a ajuda do corretor que pertence à plataforma Fog Computing. A principal responsabilidade deste corretor é gerenciar os serviços do usuário, acompanhar as VMs e os contêineres disponíveis e delegar os pedidos aos contêineres das VMs.	

Computation Layer, esta camada compreende os nDCs (Nano Datacenters) que  executam as tarefas usando os contêineres. Ela também incorpora um controlador local a nível de nó que faz o controle de todos os contêineres no nível nDC e seu status atual. Dependendo do seu status atual, o controlador seleciona os contêineres que podem ser alocados para diferentes tarefas. Além disso, ele também executa migrações de contêiner entre as VMs. Essas migrações são realizadas para reduzir o consumo total de energia dos nDCs durante falhas inesperadas.


Na etapa de seleção e escalonamento de tarefas, foram utilizados contêineres em vez das máquinas virtuais convencionais para reduzir a sobrecarga e o tempo de resposta, bem como o consumo global de energia dos dispositivos da FOG, ou seja, nano data centers (nDCs). Em comparação com máquinas virtuais (VMs), os contêineres são instâncias de virtualização relativamente leves \cite{pahl2015containers,pahl2015containerization}. Os autores consideram como principais vantagens no uso de contêineres ao invés de máquinas virtuais características importantes que incluem leveza, desempenho, eficiência e ausência da necessidade de obtenção de instruções privilegiadas. Nesse contexto, além da arquitetura descrita, os autores apresentam uma técnica de migração em tempo real para minimizar o consumo de energia. A abordagem que desencadeia a migração dos contêineres é baseada no uso de energia dos servidores. Assim, a migração deverá iniciar sempre que a utilização do servidor exceda ou diminua considerando os limites máximos e inferiores pré definidos, respectivamente. 

A metodologia de avaliação deste trabalho foi proposta com base no impacto do CoESMS na utilização global de energia utilizando Fog Computing, ao mesmo tempo que garante níveis aceitáveis de SLA. O ambiente é implementado usando uma versão estendida do popular simulador de nuvem CloudSim, ContainerCloudSim \cite{piraghaj2017containercloudsim}. A ferramenta proporciona um ambiente contêineres em relação à computação em nuvem para modelagem e simulação. Nesta etapa é feita a comparação do esquema proposto com um esquema que não implementa o CoESMS. A Figura 2 apresenta as características do ambiente de simulação.



Os testes experimentais da simulação foram repetidos 25 vezes em um período de simulação de 24 horas. Para avaliar os dois esquemas com base na configuração de simulação acima, foram utilizadas cargas de trabalho com base no PlanetLab \cite{chun2003planetlab}. O desempenho dos dois esquemas foi comparado utilizando as seguintes quatro métricas de desempenho: número total de migrações de contêineres, análise de sobrecarga de contêineres, número médio de violações de SLA e total de energia consumida de (kWh), as figuras 3 e 4 apresentam respectivamente os resultados apresentados.



De acordo com os autores, os contêineres são entidades virtuais mais leves do que VMs e suportam migrações internas e externas contínuas, sem impor nenhuma sobrecarga significativa na CPU e na utilização da memória do subjacente e na largura de banda da rede. Não obstante, os resultados apresentados demonstram que o esquema proposto reduz o consumo de energia e o número médio de violações de SLA em 21,75\% e 11,82\%, respectivamente. Os resultados podem ser atribuídos ao fato de que o esquema proposto mapeia as tarefas para os contêineres com base na abordagem multi-objetivo da teoria dos Jogos, que formula o jogo entre os contêineres e seleciona a tarefa considerada a melhor para a execução. Quando a melhoria na eficiência energética pode ser atribuída ao escalonamento de tarefas com eficiência energética em contêineres e suas migrações.

\section{Qualidade de Serviço}

Em \cite{suto2015energy}, Suto et al. propuseram um esquema de operação de um sistema de computação sem fio (WCS) baseado em Fog Computing eficiente em termos de energia e atraso de pacotes. Neste trabalho os autores consideram que existe uma relação de compensação entre o consumo de energia e o atraso para a coleta de dados, assim, o esquema proposto controla o tempo de desligamento das antenas e a conectividade de rede para reduzir o consumo de energia do sistema enquanto satisfazem um atraso de pacotes aceitável, o qual é decidido com base na exigência de aplicações IoT industriais. A adoção da Internet das coisas em uma uma fábrica inteligente introduz uma nova revolução na era industrial, onde serviços como um sistema de computação coletam vários tipos de dados de máquinas e sensores podem extrair uma grande quantidade de dados coletados obtendo informações valiosas em tempo real para a operação da fábrica \cite{salvadori2009monitoring,ovsthus2014industrial}. Assim, torna-se possível otimizar a operação da fábrica sem a intervenção direta de recursos humanos. 

Neste cenário, os autores consideraram que, em comparação com os serviços providos diretamente pela nuvem, o uso de Fog computing - onde os nós de computação estão localizados fisicamente mais pŕoximos dos sensores para se comunicar diretamente com os nós do gateway - a rede baseada em Fog pode reduzir drasticamente a latência do feedback. Não obstante, a fim de obter capacidade suficiente para fornecer esse serviço, um nó de computação de nevoeiro deve satisfazer requisitos como, alto desempenho de processamento para suportar a grande exploração de dados em tempo real, coleta de dados simultânea de muitos sensores, alta disponibilidade de serviços e baixo consumo de energia do sistema para uma operação de fábrica de baixo custo.

Com o objetivo de atender a estes critérios, este trabalho se concentra em um Sistema de Computação Sem Fio  (WCS) \cite{shin2013feasibility,suto2015failure}, uma vez que o WCS efetivamente acomoda muitos servidores, ele pode alcançar uma alta capacidade de processamento, mesmo que o espaço seja limitado. Além disso cada nó foi equipado com antenas de 60 GHz onde a alta taxa de transmissão de dados - entre  4 e 15 Gbps permite coletar dados de muitos sensores simultaneamente.

Uma vez que o serviço IoT é sensível ao atraso, faz-se necessário satisfazer um atraso aceitável para a coleta de dados. Além disso, em um ambiente industrial é uma tarefa primordial reduzir o consumo de energia para alcançar uma fábrica eficiente de energia. Assim, para abordar esta questão de pesquisa, os autores construíram um modelo matemático para avaliar o consumo de energia do sistema e o atraso na coleta de dados, e assim mostrar uma relação de troca entre esses valores. O modelo apresentado considera que para executar o controle de feedback em tempo real, o sistema de computação coleta vários tipos de dados de sensores que são implantados em toda a fábrica, que denotam o conjunto de tipos de dados. Além disso, uma vez que o sistema de computação possui conhecimento sobre dinâmicas de dados (ou seja, tempo de chegada de dados) e o atraso interno aceitável de cada dado é previamente definido, ele conhece o atraso interno aceitável do tipo de dados.

Além disso, com base no trade-off, os autores propuseram um sistema de computação sem fio eficiente em energia e atraso (E2DA-WCS). Na arquitetura proposta, a proporção de servidores em estado de suspensão e modo dos servidores em estado ativo em cada intervalo de tempo são controlados dinamicamente para minimizar o consumo de energia do sistema enquanto satisfazem um atraso interno aceitável.

A avaliação de desempenho do esquema proposto considerando eficiência energética e atraso foi previsto usando simulações computacionais. O cenário de avaliação considera um ambiente com constituído por 100 servidores executando a simulação durante 100 segundos. Além disso, a perda do caminho é calculada usando as equações descritas em \cite{suto2015failure}. Para verificar a eficácia a proposta apresentada a mesma foi comparada os esquemas convencionais. Como um dos esquemas convencionais, os autores utilizaram um esquema que controla dinamicamente o tempo de desligamento, mas usa um grau constante, conhecido como sono dinâmico com grau constante (DSCD). O outro esquema, referido como Sono constante com grau constante (CSCD), usa consistentemente um número constante de servidores no estado de suspensão e o grau de servidores no estado ativo. As figuras I e II apresentam os resultados obtidos.

A figura I apresenta a quantidade de consumo de energia do sistema por um período de 100 segundos. O consumo de energia do sistema é calculada como a soma da potência consumida no instante t de 0 a 100 segundos. Conforme mostrado, no caso do cenário 1, proposta atinge aproximadamente 13 e 10 por cento de redução no consumo de energia do sistema em comparação com DSCD e CSCD, respectivamente. Além disso, a figura II demonstra o índice de satisfação de atraso nos diferentes cenários. A partir deste resultado, os autores concluem que a metodologia apresentada pode atingir a proporção máxima no ambiente de simulação. Por outro lado, o DSCD atinge cerca de 60 por cento do índice de satisfação no cenário 1 e a proporção diminui ainda mais no cenário 2. A partir desse fenômeno, podemos notar que o grau constante não pode satisfazer o atraso aceitável e o controle conjunto do tempo de desligamento e o estado dos servidores tem um grande impacto na relação de satisfação.
Neste trabalho, os autores investigaram um novo sistema de computação para fábricas inteligentes com IoT e Fog Computing. Além disso, criaram um modelo matemático para avaliar o desempenho do sistema proposto com base na complexa teoria da rede. Este modelo mostrou a existência de uma relação de trade-off entre consumo de energia e atraso necessário para a coleta de dados. Consequentemente, foi proposto um esquema de operação eficiente em termo de consumo de energia 
e atraso de pacotes.

\section{Arquiterura verde}

 Sakar et al. \cite{sarkar2016theoretical}, apresentam um modelo teórico para arquitetura de computação em nevoeiro e compara seu desempenho com o modelo tradicional de computação em nuvem. Este trabalho foi motivado pela necessidade de disponibilizar serviços de baixa latência em tempo real simultaneamente para bilhões de dispositivos IoT na borda da rede diante da arquitetura tradicional da computação em nuvem. É importante observar que estes dois paradigmas não substituem um ao outro, e sim se complementam.
    Nesta direção, os autores propuseram uma formulação matemática para o paradigma computacional “Fog Computing”, definindo seus componentes individuais e apresentando um estudo comparativo com a computação em nuvem em termos de latência de serviço e consumo de energia. O modelo apresentado
As métricas definidas neste trabalho para análise de desempenho da computação em nevoeiro em comparação com o modelo de computação em nuvem tradicional são a latência de serviço e consumo de energia. A latência do serviço é o tempo de resposta para uma solicitação enviada por uma instância do aplicativo que está sendo executada dentro de um dispositivo IoT, esta métrica é ‘calculada como a soma da latência de transmissão e a latência de processamento para a solicitação. Por outro lado, o consumo de energia é medido como a energia gasta durante à transmissão de bytes da unidade de dados do nível inferior para a camada intermediária, e do nível intermediário para a camada superior, ou seja, todo o caminho que os dados percorrem entres os dispositivos IoT, passando pela camada de computação em nevoeiro até a nuvem.
O cenário do experimento considera a velocidade de processamento dos dispositivos no nível de computação em nevoeiro equivalentes ao processador ARM Cortex A5 e os centros de dados da nuvem são tomados como processadores Intel Core i7 4770k respectivamente.
Segundo os autores, os resultados mostram que, para um cenário em que 25\% dos aplicativos IoT exigem serviços em tempo real e de baixa latência, o gasto médio de energia na computação em nevoeiro é 40,48\% menor do que o modelo convencional de computação em nuvem. A partir desta análise de desempenho, os autores discutem a aplicabilidade da vida real do paradigma da computação em nevoeiro, juntamente com a estrutura tradicional de computação em nuvem, e algumas de suas implementações práticas. Observou-se que, para um sistema com grande número de aplicações IoT em tempo real e de baixa latência, a latência de serviço associada à computação em nevoeiro foi significativamente menor que a da computação em nuvem.



\section{Otimização de Microprocessadores}

Em \cite{nassiffe2016optimising}, Nassiffe et al. apresentam um mecanismo para selecionar dinamicamente a freqüência da CPU para executar tarefas considerando restrições de escalabilidade e disponibilidade de energia. O mecanismo proposto leva em consideração um cenário onde a carga computacional das tarefas não é constante, mas varia de forma dinâmica ao longo do tempo.  Segundo os autores, é crescente o número de sistemas embarcados que dependem de baterias para operar. Tais sistemas são encontrados em muitas aplicações, entre elas, entretenimento, saúde, agricultura e etc. Além disso, estes devem ser capaz de tomar decisões em ambientes desestruturados, ou seja, onde as condições ambientais não podem ser previamente determinadas ou que realizem missões que possam pôr em perigo a vida humana ao mesmo tempo em que o sistema conseve energia e mantenha níveis aceitáveis de QoS. \cite{nassiffe2016optimising,raibert2008bigdog}. 
Por outro lado, a escala dinâmica de tensão e frequência (DVFS) é uma técnica comumente utilizada para obter economias de energia em sistemas em tempo real. Diversos trabalhos utilizam esta técnica na tentativa de economizar energia em sistemas embarcados \cite{saha2012experimental,snowdon2005power,zhu2004effects,aydin2006system}, contudo, os autores alegam que nem sempre questões como, outros dispositivos além da CPU e QoS nem sempre são considerados nos trabalhos. Portanto, neste trabalho, a metodologia apresentada parte da necessidade de ajustar a freqüência de acordo com o tempo necessário para realizar cada tarefa, a carga da CPU e a energia disponível considerando que que o sistema ofereça serviços com a melhor qualidade possível.

O mecanismo desenvolvido baseia-se em métodos de ponto interior utilizados como um meio de otimizar a QoS do sistema sujeito a restrições de energia e agendamento. O problema é formulado em programação matemática e resolvido com um algoritmo de otimização convexa onde supõe-se que a frequência da CPU pode ser selecionada continuamente dentro de um determinado intervalo. Tais métodos são conhecidos por serem computacionalmente eficientes. Como o problema de reconfiguração definido é convexo, ele pode ser resolvido usando o método do ponto interior que garante a convergência para o ótimo global \cite{boyd2004convex}. Este método pode ser visto como uma estrutura para problemas convexos, pelo que o método de Newton é aplicado a uma seqüência de aproximações sem restrições. 

A eficácia do mecanismo de reconfiguração proposto foi apresentado através de uma análise numérica do algoritmo. Assim, o desempenho da abordagem foi avaliado verificando o número de iterações necessárias para resolver problemas divididos em fases. Para isso, foram criados 1000 conjuntos de tarefas sintéticas com 10, 30 e 50 tarefas cada um usando o algoritmo UUniFast \cite{bini2005measuring}. O diagrama de caixa  apresenta os resultados obtidos no experimento.

A análise apresentado indica que quando o número de tarefas cresce, o algoritmo torna-se mais sensível aos parâmetros que definem a instância do problema, onde o número de iterações os tende a variar mais com 50 tarefas do que 10 ou 30. Por outro lado, o número de iterações para resolver o problema não aumentou na mesma taxa do número de tarefas considerando que, quando o conjunto de tarefas aumenta de 10 a 50 tarefas, 500 por cento , o número de iterações a resolver aumentou 178 por cento. Além disso, os autores consideram que uma solução viável pode ser conseguida resolvendo apenas uma das fases.


Por outro lado, Zahaf e Houssam-Eddine et al. \cite{zahaf2017energy} propõem uma heurística para paralelização e alocação de threads em plataformas multicore heterogêneas ajustando dinamicamente os níveis de energia dos núcleos com o objetivo de reduzir o consumo total de energia sem comprometer o sistema. Neste trabalho os autores apresentam um modelo de consumo de energia e desempenho para execução de tarefas paralelas em tempo real para a arquitetura ARM big.LITTLE. O consumo de energia é considerado um fator de extrema importância em sistemas computacionais de tempo real, principalmente quando operados com a energia de bateria. Este cenário se agrava quando os sistemas distribuídos em tempo real integrados devem suportar aplicações cada vez mais complexas coletando uma grande quantidade de dados de sensores. Um cenário típico para este tipo de aplicação é a Fog Computing, ou computação em nevoeiro, onde uma certa quantidade de dados coletada do meio ambiente, precisa ser pré-processada em tempo real antes de ser enviada para o servidor central para armazenamento e realizar processamento final. Segundo \cite{zahaf2017energy}, muitas dessas aplicações podem ser facilmente paralelizadas distribuindo dados através dos elementos de computação paralela. Por outro lado, a tecnologia de multicore heterogênea pode ajudar a alcançar a pontualidade e o baixo consumo de energia, mesmo quando a carga computacional não é muito alta, pois neste cenário, os processadores são mais eficientes que uma plataforma equivalente de um único núcleo \cite{wolf2012computers}. A ideia é operar o sistema em uma "freqüência mais baixa", definir alguns núcleos em um estado de energia profunda e, ao mesmo tempo, reduzir o tempo de resposta da tarefa ao decompor uma tarefa esporádica em threads paralelos para serem divididos em todos os núcleos ativos.
Modelos de tarefas paralelas de tempo real são classificadas de acordo com o nível de paralelismo (rígido, moldável e maleável), de acordo com, \cite{drozdowski2004scheduling,dutot2004scheduling} a maioria das aplicações paralelas no mundo real são moldáveis. Considerando estas questões, os autores se concentraram em tarefas moldáveis em tempo real. A fim de obter uma decomposição ideal em relação ao consumo de energia para um conjunto de tarefas na plataforma multi-core heterogênea, é necessário definir a freqüência de operação dos núcleos, decompor cada tarefa em um conjunto de threads paralelos e realizar uma análise de planejamento e alocar os segmentos para os núcleos para garantir que cada segmento seja concluído antes do prazo.



Por outro lado, os autores utilizam um modelo, onde definem o problema de otimização como um problema de Programação Não-linear Inteira (INLP). O problema da alocação de tarefas sem decomposição já é um problema NP completo, adicionando a decomposição e a seleção de freqüência, o problema torna-se mais difícil e não linear. Dadas essas definições, os autores apresentam uma formulação do problema como problema INLP. A função objetivo apresentada neste trabalho considera a soma de dois termos. O primeiro termo expressa a energia que depende da execução da thread enquanto o segundo termo expressa o consumo de energia comum que é gerado por pelo menos um segmento quando alocado. 
Além disso, os autores propõem uma heurística de escalonamento baseada em um algoritmo guloso para selecionar as freqüências e os pontos de corte e para alocar as threads nos diferentes núcleos disponíveis, de modo a minimizar o consumo total de energia, garantindo que todos os prazos sejam respeitados.
Para avaliar a abordagem, apresentam uma ampla gama de experiências sintéticas que demonstram o desempenho da abordagem proposta. A heurística foi aplicada em um conjuntos de tarefas sintéticas geradas aleatoriamente testados em dois cenários com base no método UUniFast \cite{davis2009priority}. A plataforma de experimento adotada consiste em placas ODROID XU32 compostas por um Samsung Exynos 5422, um GPU Mali, memória RAM e periféricos de E/S. Em cada experimento, foram alocados os segmentos de avaliação comparativo nos núcleos operando em uma freqüência fixa. Em cada cenário variou-se, o número de threads, o núcleo em que o segmento é alocado e a freqüência de operação desse núcleo. A figura 7 apresenta o consumo de energia para pequenos e grandes núcleos.


%\documentclass[twocolumn]{article}



Os resultados apresentados na figura I, traçam o consumo de energia de núcleos pequenos e grandes em função da utilização total. Para pequenos núcleos, o CPM consome mais energia do que as heurísticas bin-packing, ou ou problema do empacotamento. No entanto, esta energia é recuperada em núcleos grandes, onde o CPM consome menos energia que a heurísticas BF. Não obstante, os autores destacam como trabalhos futuros a implementação de técnicas de DVFS para tirar proveito desta característica e expandir o modelo de tarefa di-gráfico proposto por Stigge et al. em \cite{stigge2011digraph}, que é projetado para expressar esse tipo de tarefas dinâmicas.


\section{Considerações Finais do capítulo}

% Please add the following required packages to your document preamble:
% \usepackage{booktabs}
% \usepackage{graphicx}
\begin{table*}
\centering
\caption{Visão geral de alguns aspectos da revisão bibliográfica}
\label{my-label}
\resizebox{\textwidth}{!}{%
\begin{tabular}{@{}llclll@{}}
\toprule
Ano (Autor)           & Metodologia                                    & \multicolumn{1}{l}{Considera o nível de bateria?} & Técnica / Algoritmo    & Métricas                                 & Aplicação                               \\ \midrule
2015 (Suto et al.)    & Desligamento das antenas e comunicação         & Não                                               & Complex network theory & SLA e consumo de energia                 & Simulação (plataforma não especificada) \\
2016 (Nassife et al.) & Variação de fequência de CPU (DVFS)            & Sim                                               & UUniFast               & Consumo de energia                       & Simulação (Análise numérica)            \\
2016 (Sackar et al.)  & Modelo matemático                              & Não                                               & Não se aplica          & Latência de serviço e consumo de energia & Simulação (plataforma não especificada) \\
2017 (KAur et al.)    & Seleção e migração de tarefas                  & Não                                               & Teoria dos Jogos       & SLA e consumo de energia                 & Simulação (ContainerCloudSim)           \\
2017 (Zahaf et al.)   & Paralelização de tarefas e alocação de threads & Sim                                               & Bin packing            & Consumo de energia                       & Prototipagem (ARM big.LITTLE)           \\ \bottomrule
\end{tabular}%
}
\end{table*}
%\FloatBarrier

Como a computação de nevoeiro é um paradigma relativamente recente, a pesquisa ainda está em seu estágio inicial. O trabalho em curso elenca desafios que podem aprimorar as capacidades de gerenciamento de recursos e eficiência energética no contexto do IoT. Nesta direção, os trabalhos discutem a definição da arquitetura \cite{sarkar2016theoretical} e a avaliação da sua adequação no contexto de IoT \cite{gupta2016ifogsim}. Outros aspectos incluem alocação de recursos \cite{kaur2017container}, qualidade de serviço \cite{suto2015energy}, análise de dados em tempo real e consciente do contexto [9], bem como um modelo de otimização de microprocessadores \cite{nassiffe2016optimising} e escalonamento de tarefas para obter desempenho mantendo a eficiência energética \cite{zahaf2017energy}.

Em \cite{sarkar2016theoretical} observou-se que a latência de serviço associada à computação em nevoeiro foi significativamente menor que a da computação em nuvem. Não obstante, o trabalho não discutiu questões como gerenciamento de recursos e métodos de virtualização. 
Kaur et al. \cite{kaur2017container} propuseram uma arquitetura que utiliza métodos de seleção e escalonamento de tarefas para reduzir o consumo de energia ao mesmo tempo em que mantém requisitos de SLA. A utilização de contêiners mostrou-se promissora, contudo, o trabalho limitou-se a implementação em ambientes de simulação.  O trabalho apresentado por \cite{suto2015energy},  propõe um esquema que controla o tempo de desligamento das antenas e a conectividade de rede para reduzir o consumo de energia do sistema enquanto satisfazem níveis aceitáveis de SLA. 
Por outro lado, o trabalho proposto do Nassife \cite{nassiffe2016optimising}, apresenta um mecanismo utilizando métodos de ponto interno como meio para otimizar a QoS de sistemas de tempo real sujeitos a restrições de energia e escalonamento. Seu método mostrou-se promissor, contudo, a pesquisa não contemplou tarefas com mais de um modo de operação e valores discretos para a variação de frequência da CPU. A metodologia apresentada em \cite{zahaf2017energy} paraleliza e aloca threads em plataformas multicore heterogêneas, sua abordagem é eficaz quando comparada aos algoritmos clássicos de heurística bin-packing. Todavia, mecanismos para economia de energia que consideram o tempo ocioso do processador não foram apresentadas. 
    
Os dados apresentados da tabel I mostram que, embora muita pesquisa tenha se concentrado nos desafios relacionados à natureza e comunicação da IoT, os desafios da em relação aos microprocessadores e otimização dos dispositivos receberam muito menos atenção. Além disso,  embora a maior parte dos trabalhos apresentados discutam questões de consumo de energia, nem todos consideraram os víveis de bateria dos dispositivos. 
\chapter{Conclusão}

\lipsum

\bibliography{Bibliografia}

% ----------------------------------------------------------
% ELEMENTOS PÓS-TEXTUAIS
% ----------------------------------------------------------
\postextual

\renewcommand{\chapnumfont}{\chaptitlefont}
\renewcommand{\afterchapternum}{}
\begin{apendicesenv}

% Imprime uma página indicando o início dos apêndices
\partapendices

% ----------------------------------------------------------
\chapter{Quisque libero justo}
% ----------------------------------------------------------

\lipsum[50]

% ----------------------------------------------------------
\chapter{Nullam elementum urna vel imperdiet sodales elit ipsum pharetra ligula
ac pretium ante justo a nulla curabitur tristique arcu eu metus}
% ----------------------------------------------------------
\lipsum[55-57]

\end{apendicesenv}

\begin{anexosenv}


% Imprime uma página indicando o início dos anexos
\partanexos

% ---
\chapter{Morbi ultrices rutrum lorem.}
% ---
\lipsum[30]

% ---
\chapter{Cras non urna sed feugiat cum sociis natoque penatibus et magnis dis
parturient montes nascetur ridiculus mus}
% ---

\lipsum[31]

% ---
\chapter{Fusce facilisis lacinia dui}
% ---

\lipsum[32]


\end{anexosenv}


\end{document}